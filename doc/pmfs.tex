\documentclass{article}
\usepackage{amsmath}
\usepackage[a4paper,total={8.5in,11.5in},centering]{geometry}
\usepackage{array}
\usepackage{calc}
\begin{document}

\renewcommand{\theequation}{\Alph{equation}}

\newcommand{\refs}{\textbf{Equation references:}}

\newcommand{\rismprog}{\textbf{Output of \texttt{rism0}:}}

% suppress page number
\pagenumbering{gobble}

\begin{center}
\begin{tabular}{ >{\arraybackslash}m{2.5in}  >{\arraybackslash}m{2.2in}  >{\arraybackslash}m{2.5in} }

Formula
&
Meaning
&
Notes
\\
\hline

{
\begin{align}
  \beta u
&= \beta u_\mathrm{LJ} + \beta u_\mathrm{c}
\notag \\
&= -\phi^* - \phi.
\label{eq:udecomp}
\end{align}
}
&
Pair potential

\refs

Eq. \eqref{eq:udecomp}: (23) in HPR1982.
&
The Coulombic potential $\phi = -\beta u_\mathrm{c}$
is written as $\phi''$ in PK1985.

The Lennard-Jones potential $u_\mathrm{LJ}$
is commonly written as $u^*$.

\rismprog

$\beta u$: Column 6 (denoted as \texttt{\$6} below).

$\beta u_\mathrm{c}$: \texttt{\$9}.

$\beta u_\mathrm{LJ}$: \texttt{\$6 -\$9}.
\\
%\hline



{
\begin{align}
\beta W
&= -\ln g
   \label{eq:Wlng} \\
&= \beta u - t. \; \mbox{(for HNC)}
   \label{eq:Wut}
\end{align}
}
&
Full potential of mean force (PMF).

\refs

Eq. \eqref{eq:Wlng}:
(45) in HRP1983,
(7) in PR1986,
(17) in PK1985.

Eq. \eqref{eq:Wut}:
(46) in HRP1983,
(8) in PR1986.
&
This notation sometimes means
the uncorrected PMF directly from
solving integral equations in RISM,
or the empirically corrected version.

The indirect correlation function $t = \tau + Q$
in XRISM.

\rismprog

Eq. \eqref{eq:Wut}: \texttt{\$6 - \$3}.
\\
%\hline



{
\begin{align}
\beta W^\mathrm{ex}
&= \beta W - \beta u
  \notag \\
&= - t \; \mbox{(for HNC)}
  \label{eq:Wexnt} \\
&= \beta \mu_{ab} - \beta \mu_a - \beta \mu_b.
  \label{eq:Wexmu}
\end{align}
}
&
Excess potential of mean force.

\refs

Eq. \eqref{eq:Wexmu}: (1) of HK2000.
&

\rismprog

$-t(r)$: \texttt{$-$\$3}.

Eq. \eqref{eq:Wexmu}: Use \texttt{chempot.py}.

\\
%\hline



{
\begin{align}
\beta W_s
&= \beta W - \dfrac{\beta u_\mathrm{c}} {\epsilon_\mathrm{RISM}}
\notag \\
&= \beta W + \dfrac{\phi} {\epsilon_\mathrm{RISM}}.
\label{eq:Wsphi}
\end{align}
}
&
The short-range part of the PMF.

\refs

Eq. \eqref{eq:Wsphi}: (47) in HRP1983.
&
Here, $\beta W$ means the PMF directly from solving RISM.
The long-range part $\beta u_\mathrm{C}/\epsilon_\mathrm{RISM}$
is subtracted.

\rismprog

$\beta W_s$: $\mathtt{\$10 + (\$6 - \$9)}$,
see Eq. \eqref{eq:deltaW}.
\\
%\hline



{
\begin{align}
\beta W^\mathrm{corr}
&= \beta W_s + \dfrac{\beta u_\mathrm{c}} {\epsilon}
    \notag \\
&= \beta W
- \dfrac{\beta u_\mathrm{c}} {\epsilon_\mathrm{RISM}}
+ \dfrac{\beta u_\mathrm{c}} {\epsilon}.
  \label{eq:Wcorrmp}
\end{align}
}
&
The empirically corrected PMF for RISM.

\refs

Eq. \eqref{eq:Wcorrmp}:
(20) in PK1985,
(11) in PR1986.
&
Here, $\beta W$ means the PMF directly from solving RISM;
$\epsilon$ is the experimental dielectric constant;
$\epsilon_\mathrm{RISM}$ is the dielectric constant from RISM.

\rismprog

$\beta W^\mathrm{corr}$:
$\mathtt{\$10} + (\mathtt{\$6} - \mathtt{\$9}) + \mathtt{\$9}/\epsilon$.
\\
%\hline


{
\begin{align}
\beta W_c
&= \beta u_\mathrm{LJ} + \dfrac{\beta u_\mathrm{c}} {\epsilon}
\notag \\
&= -\phi^* - \phi/\epsilon.
\label{eq:Wcphi}
\end{align}
}
&
The continuum part of the PMF.

\refs

Eq. \eqref{eq:Wcphi}: (49) in HRP1983.
&
\rismprog

$\beta W_c$:
$\mathtt{\$6} - \mathtt{\$9} + \mathtt{\$9}/\epsilon$.
\\
%\hline



{
\begin{align}
\beta \delta W
&= \beta W^\mathrm{corr} - \beta W_c
\notag \\
&= \beta W_s - \beta u_\mathrm{LJ}
\label{eq:deltaW} \\
&= \beta W^\mathrm{ex}
+ \beta u_\mathrm{c}
\left( 1 - \frac{ 1 } { \epsilon_\mathrm{RISM} } \right).
\label{eq:deltaWWex}
\end{align}
}
&
The difference from the corrected PMF, $\beta W^\mathrm{corr}$
and the continuum part, $\beta W_c$.

\refs

Eq. \eqref{eq:deltaW}: (50) in HRP1983.
&
\rismprog

$\beta \delta W$:
$\mathtt{\$10}$.
\\


{
\begin{align}
\beta \Delta W
&= \beta W - \beta u_\mathrm{LJ}
\notag \\
&= \beta u_\mathrm{c} + \beta W^\mathrm{ex}
\notag \\
&= \beta u_\mathrm{c} - t. \; \mbox{(for HNC)}
\label{eq:DeltaW}
\end{align}
}
&
\refs

Eq. \eqref{eq:DeltaW}: (9) in PR1986.
&
\rismprog

$\beta \Delta W$:
$\mathtt{\$9} - \mathtt{\$3}$.
\\


{
\begin{align}
\beta \Delta W'
&= \beta \Delta W - \dfrac{ \beta u_\mathrm{c} } { \epsilon_\mathrm{RISM} }
                  + \dfrac{ \beta u_\mathrm{c} } { \epsilon }
\label{eq:DeltaWp} \\
&= \beta W^\mathrm{ex} + \beta u_\mathrm{c}
                  - \dfrac{ \beta u_\mathrm{c} } { \epsilon_\mathrm{RISM} }
                  + \dfrac{ \beta u_\mathrm{c} } { \epsilon }
\notag \\
&= \beta \delta W + \dfrac{ \beta u_\mathrm{c} } { \epsilon }.
\label{eq:DeltaWpdeltaW}
\end{align}
}
&
\refs

Eq. \eqref{eq:DeltaWp}: (10) in PR1986.
&
The last step follows from Eq. \eqref{eq:deltaWWex}.

\rismprog

$\beta \Delta W'$:
$\mathtt{\$10} + \mathtt{\$9}/\epsilon$,
see Eq. \eqref{eq:DeltaWpdeltaW}.

\\
\end{tabular}
\end{center}

\end{document}
